\section{Протокол физбоя \texttt{(physsummer-battle)}}

Шаблон для ведения лога физбоя, предоставляемый пакетом, является \texttt{tikz} картинкой.

\subsection{Двойной физбой}

Пакетом предоставляется следующий набор команд.

Заголовок
\begin{lstlisting}[gobble = 3]
    \pdtitle{название команды I}{название команды II}
\end{lstlisting}
Команда для вызова
\begin{lstlisting}[gobble = 3]
    \pdcall{тип вызова}{номер задачи}{баллы команды I}{баллы команды II}
        {игрок I}{игрок II}
\end{lstlisting}
Команда для подведения итогов вручную
\begin{lstlisting}[gobble = 3]
    \pdsum{результат команды I}{результат команды II}{результат жюри}
\end{lstlisting}
и автоматически
\begin{lstlisting}[gobble = 3]
    \pdsumA
\end{lstlisting}



В пакете реализовано четыре типа вызова, представленные ниже.

\vspace{0.2cm}

\noindent
\begin{tabular}{S{0.82\linewidth} S{0.145\linewidth}}
    команда I вызывает команду II &
        \begin{lstlisting}[style = listtable, gobble = 7, keepspaces = \false]
            $->$
        \end{lstlisting} \\
    команда II вызывает команду I &
        \begin{lstlisting}[style = listtable, gobble = 7, keepspaces = \false]
            $<-$
        \end{lstlisting} \\
    команда I вызывает команду II, но получила проверку корректности &
        \begin{lstlisting}[style = listtable, gobble = 7, keepspaces = \false]
            $->|$
        \end{lstlisting} \\
    команда II вызывает команду I, но получила проверку корректности &
        \begin{lstlisting}[style = listtable, gobble = 7, keepspaces = \false]
            $|<-$
        \end{lstlisting}
\end{tabular}



Далее приведен пример протокола.

%что-то не так
\begin{lstlisting}[gobble = 3]
    \begin{tikzpicture}
        \pdtitle{Аполлон}{7921}
        \pdcall{|<-}{4}{0}{4}{Смирных}{Грудинин}
        \pdcall{->}{3}{1}{9}{Жмудь}{Нагавкина}
        \pdcall{|<-}{6}{2}{10}{Захаров}{Грудинин}
        \pdcall{->}{5}{0}{12}{Елфимова}{Люлина}
        \pdcall{<-}{2}{0}{6}{Вронская}{Болотовский}
        \pdcall{->}{1}{3}{5}{Сахно}{Хмылёв}
        \pdsumA
    \end{tikzpicture}
\end{lstlisting}

\noindent
\begin{tikzpicture}
    \pdtitle{Аполлон}{7921}
    \pdcall{|<-}{4}{0}{4}{Смирных}{Грудинин}
    \pdcall{->}{3}{1}{9}{Жмудь}{Нагавкина}
    \pdcall{|<-}{6}{2}{10}{Захаров}{Грудинин}
    \pdcall{->}{5}{0}{12}{Елфимова}{Люлина}
    \pdcall{<-}{2}{0}{6}{Вронская}{Болотовский}
    \pdcall{->}{1}{3}{5}{Сахно}{Хмылёв}
    \pdsumA
    %\pdsum{6}{46}{20}
\end{tikzpicture}


\subsection{Тройной физбой}

Лог тройного физбоя можно вести похожим способом.

\begin{lstlisting}[gobble = 3]
    \pttitle{название команды I}{название команды II}{название команды III}
\end{lstlisting}

\begin{lstlisting}[gobble = 3]
    \ptcall{#1}{номер задачи}{баллы команды I}{баллы команды II}{баллы команды III}
        {игрок I}{игрок II}{игрок III}
\end{lstlisting}

\begin{lstlisting}[gobble = 3]
    \ptsum{результат команды I}{результат команды II}{результат команды III}{результат жюри}
\end{lstlisting}

Вместо типа вызова (поскольку в тройном физбое порядок определён после конкурса капитанов) <<\#1>>
показывает, есть ли проверка корректности если вызов обычный~--- аргумент пустой, если была проверка~---
можно написать любой аргумент, например, какую-нибудь букву.
