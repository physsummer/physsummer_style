\section{Протокол физбоя \texttt{(physsummer-battle)}}

Шаблон для ведения лога физбоя, предоставляемый пакетом, является \texttt{tikz} картинкой.

\subsection{Двойной физбой}

Пакетом предоставляется следующий набор команд.

\begin{lstlisting}[gobble = 3]
    \pdtitle{название команды I}{название команды II}
\end{lstlisting}

\begin{lstlisting}[gobble = 3]
    \pdcall{тип вызова}{номер задачи}{баллы команды I}{баллы команды II}
        {игрок I}{игрок II}
\end{lstlisting}

\begin{lstlisting}[gobble = 3]
    \pdsum{результат команды I}{результат команды II}{результат жюри}
\end{lstlisting}

Ниже приведен пример протокола.

%что-то не так
\begin{lstlisting}[gobble = 3]
    \begin{tikzpicture}
        \pdtitle{Аполлон}{7921}
        \pdcall{|<-}{4}{0}{4}{Смирных}{Грудинин}
        \pdcall{->}{3}{1}{9}{Жмудь}{Нагавкина}
        \pdcall{|<-}{6}{2}{10}{Захаров}{Грудинин}
        \pdcall{->}{5}{0}{12}{Елфимова}{Люлина}
        \pdcall{<-}{2}{0}{6}{Вронская}{Болотовский}
        \pdcall{->}{1}{3}{5}{Сахно}{Хмылёв}
        \pdsum{6}{46}{20}
    \end{tikzpicture}
\end{lstlisting}

\noindent
\begin{tikzpicture}
    \pdtitle{Аполлон}{7921}
    \pdcall{|<-}{4}{0}{4}{Смирных}{Грудинин}
    \pdcall{->}{3}{1}{9}{Жмудь}{Нагавкина}
    \pdcall{|<-}{6}{2}{10}{Захаров}{Грудинин}
    \pdcall{->}{5}{0}{12}{Елфимова}{Люлина}
    \pdcall{<-}{2}{0}{6}{Вронская}{Болотовский}
    \pdcall{->}{1}{3}{5}{Сахно}{Хмылёв}
    \pdsum{6}{46}{20}
\end{tikzpicture}


\subsection{Тройной физбой}

Лог тройного физбоя можно вести похожим способом.

\begin{lstlisting}[gobble = 3]
    \pttitle{название команды I}{название команды II}{название команды III}
\end{lstlisting}

\begin{lstlisting}[gobble = 3]
    \ptcall{тип вызова}{номер задачи}{баллы команды I}{баллы команды II}{баллы команды III}
        {игрок I}{игрок II}{игрок III}
\end{lstlisting}

\begin{lstlisting}[gobble = 3]
    \ptsum{результат команды I}{результат команды II}{результат команды III}{результат жюри}
\end{lstlisting}