\section{Рисование картинок \texttt{(physsummer-tikz)}}

В пакете собран инструментарий для рисования картинок при помощи \texttt{tikz}.

\subsection{Некоторые стандартные возможности \texttt{tikz}}

\subsubsection{Углы}

Ниже приведен пример отрисовки углов. Для задания угла необходимо имееть три точки, задаваемые при помощи
\texttt{coordinate} или \texttt{node}.

На третьем рисунке приведен пример того, что можно получить координату точки, где должна появиться
подпись угла, что может быть полезно, если необходимо залить область под подписью.

\begin{minipage}{0.28\linewidth}
    \begin{tikzpicture}[mstyle, shift = {(-1, 0)}]
        \draw (2, 0) coordinate (a) -- (0, 0) coordinate (o)
            -- (60:2) coordinate (b);
        \pic[pic text = $\alpha$, draw = blue, angle eccentricity = 1.8, angle radius = 0.5cm]
            {angle = a--o--b};
        
        \draw (2, -6) coordinate (a1) -- ++(-2, 0) coordinate (o1)
            -- ++(60:2) coordinate (b1);
        \pic[pic text = $\beta$, style = double, draw = blue, angle eccentricity = 1.8, angle radius = 0.5cm]
            {angle = a1--o1--b1};
        
        \draw (2, -12) coordinate (a1) -- ++(-2, 0) coordinate (o1)
            -- ++(60:2) coordinate (b1);
        \pic (ver) [pic text = , style = double, draw = blue, angle eccentricity = 1.8, angle radius = 0.5cm]
            {angle = a1--o1--b1};
        \node[circle, inner sep = 0, fill = red!40] at (ver) {$\gamma$};
    \end{tikzpicture}
\end{minipage}
\begin{minipage}{0.72\linewidth}
    \begin{lstlisting}
        \begin{tikzpicture}
            \draw (2, 0) coordinate (a) -- (0, 0) coordinate (o)
                -- (60:2) coordinate (b);
            \pic[
                pic text = $\alpha$,
                draw = blue,
                angle eccentricity = 1.8,
                angle radius = 0.5cm]
                {angle = a--o--b};

            \draw (2, -6) coordinate (a1) -- ++(-2, 0) coordinate (o1)
                -- ++(60:1.2) coordinate (b1);
            \pic[
                pic text = $\beta$,
                style = double,
                draw = blue,
                angle eccentricity = 1.8,
                angle radius = 0.5cm]
                {angle = a1--o1--b1};

            \draw (2, -12) coordinate (a1)
                -- ++(-2, 0) coordinate (o1)
                -- ++(60:2) coordinate (b1);
            \pic (ver) [
                pic text = ,
                style = double,
                draw = blue,
                angle eccentricity = 1.8,
                angle radius = 0.5cm]
                {angle = a1--o1--b1};
            \node[circle, inner sep = 0, fill = red!40]
                at (ver) {$\gamma$};
        \end{tikzpicture}    
    \end{lstlisting}
\end{minipage}




\subsection{Возможности пакета}

