\subsection{Возможности пакета}

\subsubsection{Касательные $2$}

В пакете реализована возможность отрисовки касательных к простым кривым.

\begin{minipage}{0.28\linewidth}
    \begin{tikzpicture}[mstyle]
        \draw[thick, tangent = 0.1, tangent = 0.8] (0, 0) to[out = 0, in = 180] (4.5, 7);
        \draw[thick, blue, use tangent, ->] (0, 0) -- (1, 0);
        \fill[blue] (tangent point-1) circle (0.07) node[black, below] {$A$};
        \draw[thick, blue, use tangent = 2, ->] (0, 0) -- (1, 0);
        \fill[blue] (tangent point-2) circle (0.07) node[black, below right] {$B$};
    \end{tikzpicture}
\end{minipage}
\begin{minipage}{0.72\linewidth}
    \begin{lstlisting}[gobble = 7]
        \begin{tikzpicture}
            \draw[thick, tangent = 0.1, tangent = 0.8]
                (0, 0) to[out = 0, in = 180] (4.5, 5);
                % задание кривой                   
                % и касательных на расстояниях 0.1 и 0.8
                % длины от начала кривой
                
            \draw[thick, blue, use tangent, ->] (0, 0) -- (1, 0);
                % использование касательной в первой точке
                % "use tangent"
                % (0, 0) является точкой касания
            
            \fill[blue] (tangent point-1) circle (0.07)
                node[black, below] {$A$};
                % первая точка касания (tangent point-1)
            
            \draw[thick, blue, use tangent = 2, ->] (0, 0) -- (1, 0);
                % использование касательной во второй точке

            \fill[blue] (tangent point-2) circle (0.07)
                node[black, below right] {$B$};
                % вторая точка касания (tangent point-2)
        \end{tikzpicture}
    \end{lstlisting}
\end{minipage}


\subsubsection{Поверхности и физические объекты}

В пакете представлен набор шаблонов, которые могут оказаться полезными при подготовке рисунков к
физическим задачам.

\begin{minipage}{0.28\linewidth}
    \begin{tikzpicture}[mstyle]
        \node at (0, 0.5) {};
        \draw[platform] (3, 0) -- (0, 0);

        \draw[thick, perpinterface] (1, -1) arc (90:-90:1);


        \draw[platform] (0, -5) -- (0, -4);
        \draw[spring] (0, -4.5) -- ++(3, 0);
    \end{tikzpicture}
\end{minipage}
\begin{minipage}{0.72\linewidth}
    \begin{lstlisting}[gobble = 7]
        \begin{tikzpicture}
            \draw[platform] (3, 0) -- (0, 0);
                % "мохнатая платформа"
                % корректно работает только в случае прямой линии

            \draw[thick, perpinterface] (1, -1) arc (90:-90:1);
                % "мохнатая кривая поверхность"


            
            \draw[platform] (0, -5) -- (0, -4);
            \draw[spring] (0, -4.5) -- ++(3, 0); % пружина
        \end{tikzpicture}
    \end{lstlisting}
\end{minipage}

\begin{minipage}{0.28\linewidth}
    \begin{tikzpicture}[mstyle]
        \node at (0, 0.5) {};
        \draw[water edge] (0, 0) -- ++(3, 0);

        \draw[marrow, thick] (0, -1.2) -- ++(3, 0);

        \draw[very thick, blue, mark position = 0.25(a), mark position = 0.6(b), mark position = 0.75(c)]
            (0, -6) arc (180:0:1) arc (-180:0:1);
        \fill[red] (a) circle (0.07) node[above, black] {$A$};
        \fill[red] (b) circle (0.07) node[below left, black] {$B$};
        \fill[red] (c) circle (0.07) node[above, black] {$C$};
    \end{tikzpicture}
\end{minipage}
\begin{minipage}{0.72\linewidth}
    \begin{lstlisting}[gobble = 7]
        \begin{tikzpicture}
            \draw[water edge] (0, 0) -- ++(3, 0); % граница воды


            \draw[marrow, thick] (0, -1.2) -- ++(3, 0);
            % стрелка на середине отрезка

            \draw[
                very thick,
                blue,
                mark position = 0.25(a),
                mark position = 0.6(b),
                mark position = 0.75(c)]
                % отметить точки на позициях
                % 0.25, 0.6, 0.75 длины от начала
                (0, -6) arc (180:0:1) arc (-180:0:1);
            \fill[red] (a) circle (0.07) node[above, black] {$A$};
            \fill[red] (b) circle (0.07) node[below left, black] {$B$};
            \fill[red] (c) circle (0.07) node[above, black] {$C$};
        \end{tikzpicture}
    \end{lstlisting}
\end{minipage}


Следующий паттерн для отрисовки воды может быть полезен, если обычная заливка плохо отображается на печати.

\begin{minipage}{0.28\linewidth}
    \begin{tikzpicture}[mstyle]
        \node at (0, 1) {};
        \fill[pattern = water] (0, 0) {[rounded corners = 3pt] -- (0, -2.5) -- (2, -2.5)} -- (2, 0)
            decorate[water edge]{-- cycle};
        \draw[very thin, water edge] (2, 0) -- (0, 0);
        
        \draw[very thick, rounded corners = 3pt] (0, 0.5) -- (0, -2.5) -- (2, -2.5) -- (2, 0.5);
    \end{tikzpicture}
\end{minipage}
\begin{minipage}{0.72\linewidth}
    \begin{lstlisting}[gobble = 7]
        \begin{tikzpicture}
            \fill[pattern = water] (0, 0)
                {[rounded corners = 3pt] -- (0, -2.5) -- (2, -2.5)}
                -- (2, 0) decorate[water edge]{-- cycle};
            \draw[very thin, water edge] (2, 0) -- (0, 0);

            \draw[very thick, rounded corners = 3pt] (0, 0.5) --
                (0, -2.5) -- (2, -2.5) -- (2, 0.5);
        \end{tikzpicture}
    \end{lstlisting}
\end{minipage}


\subsubsection{Электрические схемы}

Пакет поддерживает библиотеку \texttt{circuit.ee.IEC} и несовместим с библиотекой \texttt{circuittikz}. В
пакете представлены шаблоны для вольтметра и амперметра.


\begin{minipage}{0.28\linewidth}
    \begin{tikzpicture}[mstyle, circuit ee IEC]
        \def\l{1.7}
        \draw (0, 0) node[contact] (A) {} to[voltmeter] ++(0, \l) node[contact] (B) {};
        \draw (A) to[voltmeter] ++(\l, 0) to[ammeter] ++(0, \l) to[battery] ++ (-\l, 0);
        \draw (A) -- ++(-0.5 * \l, 0) node[contact] {};
        \draw (B) -- ++(-0.5 * \l, 0) node[contact] {};
        \draw (-0.5 * \l, 0.5 * \l) node {$U$};
    \end{tikzpicture}
\end{minipage}
\begin{minipage}{0.72\linewidth}
    \begin{lstlisting}[gobble = 7]
        \begin{tikzpicture}[circuit ee IEC]
            \def\l{1.7}
            \draw (0, 0) node[contact] (A) {}
                to[voltmeter] ++(0, \l) node[contact] (B) {};
            \draw (A) to[voltmeter] ++(\l, 0)
                to[ammeter] ++(0, \l) to[battery] ++ (-\l, 0);
            \draw (A) -- ++(-0.5 * \l, 0) node[contact] {};
            \draw (B) -- ++(-0.5 * \l, 0) node[contact] {};
            \draw (-0.5 * \l, 0.5 * \l) node {$U$};
        \end{tikzpicture}
    \end{lstlisting}
\end{minipage}