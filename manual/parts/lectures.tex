\section{Конспекты \texttt{(physsummer-thm, physsummer-tables)}}

\subsection{Теоремы, утверждения, и т.д.}

Работа с теоремами отсуществляется при помощи пакета \texttt{tcolorbox}. Если вы хотите использовать
дополнительные инструменты, то рекомендется обратить внимание на совместимость с этим пакетом.

\begin{lstlisting}[gobble = 3]
    \begin{theorem}
        Пусть $X_1, X_2, \dots, X_n$~--- независимые случайные величины
        с мат. ожиданиями $\mathbb{E}[X_i] = \mu_i$ и дисперсиями
        $Var[X_i] = \sigma_i^2$. Тогда для любого $a > 0$:
        $$
            \Pr[\abs{\sum\limits_{i = 1}^{n} X_i - \sum\limits_{i = 1}^{n} \mu_i} \ge a] \le 
            \frac{\sum\limits_{i = 1}^n \sigma_i^2}{a^2}.
        $$
    \end{theorem}
\end{lstlisting}

\begin{theorem}
    Пусть $X_1, X_2, \dots, X_n$~--- независимые случайные величины с мат. ожиданиями $\mathbb{E}[X_i] =
    \mu_i$ и дисперсиями $Var[X_i] = \sigma_i^2$. Тогда для любого $a > 0$:
    $$
        \Pr[\abs{\sum\limits_{i = 1}^{n} X_i - \sum\limits_{i = 1}^{n} \mu_i} \ge a] \le
        \frac{\sum\limits_{i = 1}^n \sigma_i^2}{a^2}.
    $$
\end{theorem}

Пакет предоставляет следующие окружения: \texttt{definition}, \texttt{conjecture}, \texttt{lemma},
\texttt{corollary}, \texttt{prorosition}, \texttt{exersice}, \texttt{example}, \texttt{remark},
\texttt{remark\_lect} и \texttt{proof}. Работа с данными окружениями осуществляется аналогично работе с
окружением \texttt{theorem}.


\subsection{Авторские комментарии}
\label{sec:authors}

При помощи команды:
\begin{lstlisting}[gobble = 3]
    \newauthor{сокращенное имя}{подпись}{цвет фона комментариев}
\end{lstlisting}

После вызова данной команды пользвателю открывается возможность комментирования при помощи команды:

\begin{lstlisting}[gobble = 3]
    \#1comment{текст комментария}
\end{lstlisting}

При этом в качестве параметра \#1 используется сокращенное имя комментатора.


\begin{lstlisting}[gobble = 3]
    \newauthor{dov}{Довлатов}{red!15}

    \dovcomment{
        Всю жизнь я дул в подзорную трубу и удивлялся, что нету музыки.
    }
\end{lstlisting}

\vspace{0.2cm}

\todo[inline, size = \small, backgroundcolor = {red!15}]{
    1: Всю жизнь я дул в подзорную трубу и удивлялся, что нету музыки. --~\textbf{Довлатов}
}

    
Комментарии нумеруются для каждого автора отдельно. Для комментариев заводится счетчик
\texttt{\#1comment}. Комментарии оборажаются только в режиме учителя.




\subsection{Таблицы}
\label{sec:tables}

Для более эффективной реализации таблиц подключен пакет \texttt{tabularx}.


В настоящем пакете реализован ряд дополнительных типов колонок.

\vspace{0.4cm}

\noindent
\begin{longtable}{C{0.82\linewidth} S{0.145\linewidth}}
    Колонка занимает всю, предоставленную ширину, применяется только в \texttt{tabularx}. В случае
    нескольких колонок указанного типа, создаются равной ширины. Выравнивание влево. &
        \begin{lstlisting}[style = listtable, gobble = 6, keepspaces = \false]
            X
        \end{lstlisting} \\
    Аналогично \texttt{X}, но с выравниванием по центру. &
        \begin{lstlisting}[style = listtable, gobble = 6, keepspaces = \false]
            Y
        \end{lstlisting} \\
    Колонка занимает ширину \#1. Выравнивание влево. &
        \begin{lstlisting}[style = listtable, gobble = 8, keepspaces = \false]
            L{#1}
        \end{lstlisting} \\
    Колонка занимает ширину \#1. Выравнивание по центру. &
        \begin{lstlisting}[style = listtable, gobble = 8, keepspaces = \false]
            C{#1}
        \end{lstlisting} \\
    Колонка занимает ширину \#1. Выравнивание вправо. &
        \begin{lstlisting}[style = listtable, gobble = 8, keepspaces = \false]
            R{#1}
        \end{lstlisting} \\
    Колонка занимает ширину \#1. Выравнивание вправо. Текстом автоматически заполняется номером колонки
    (счетчик \texttt{trownumber}).  &
        \begin{lstlisting}[style = listtable, gobble = 8, keepspaces = \false]
            N{#1}
        \end{lstlisting} \\
\end{longtable}

Также в пакете реализована отрисовка линий указанной толщины при помощи команд:
\begin{lstlisting}[style = listtable, gobble = 3]
    \hlinewd{#1} % горизонтальная линия
    ?{#1}        % вертикальная линия
\end{lstlisting}

\vspace{0.2cm}

Ниже приведен пример испольвания. Стоит обратить внимание, что таблицы плохо взаимодействуют с отступами,
возможно полезной окажется команда \texttt{noindent}. Для изменения вертикальных отступов в таблице можно
задать параметр \texttt{arraystretch}.


\begin{lstlisting}[style = listtable, gobble = 3]
    \def\arraystretch{1.5}

    \noindent
    \begin{tabular}{?{2pt}N{0.1\linewidth}|C{0.41\linewidth}|R{0.41\linewidth}?{2pt}}
        \hlinewd{2pt}
        & текст по центру & текст справа \\
        \hline
        & $\frac{a}{b}$ & $\pdv{P}{x}$ \\
        \hlinewd{2pt}
    \end{tabular}

    \vspace{0.5cm}

    \noindent
    \begin{tabularx}{\linewidth}{|X?{2pt}Y?{2pt}c|}
        \hline
        1 & 2 & 3 \\
        \hline
        $\vec{F} = m \vec{a}$ & $\var{Q} = c m \Delta T$ &  $U = I R$ \\
        \hline
    \end{tabularx}
\end{lstlisting}

\vspace{0.5cm}
\def\arraystretch{1.5}

\noindent
\begin{tabular}{?{2pt}N{0.1\linewidth}|C{0.41\linewidth}|R{0.41\linewidth}?{2pt}}
    \hlinewd{2pt}
        & текст по центру & текст справа \\
    \hline
        & $\frac{a}{b}$ & $\pdv{P}{x}$ \\
    \hlinewd{2pt}
\end{tabular}

\vspace{0.5cm}

\noindent
\begin{tabularx}{\linewidth}{|X?{2pt}Y?{2pt}c|}
    \hline
    1 & 2 & 3 \\
  \hline
   $\vec{F} = m \vec{a}$ & $\var{Q} = c m \Delta T$ &  $U = I R$ \\
    \hline
\end{tabularx}

\def\arraystretch{-500}