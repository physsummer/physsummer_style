\section{Установка и содержимое пакета}

Пакет расположен в git репозитории bitbucket. Для для скачивания можно использовать одну из следующих
опций:
\begin{itemize}
    \item при работе в консоле достаточно выполнить команду:
        \begin{lstlisting}
            git clone https://bitbucket.org/physsummer/physsummer_style.git
        \end{lstlisting}
    \item также можно скачать вручную по ссылке:
        \begin{lstlisting}
            https://bitbucket.org/physsummer/physsummer_style/
        \end{lstlisting}
\end{itemize}


Скачать пакет можно в любую желаемую папку. После установки пакета небходимо добавить путь к папке в
настройки дистрибутива \LaTeX. Например в дистрибутиве \texttt{TexLive} для этого достаточно настроить
переменную окружения:

\begin{itemize}
    \item в операционной системе \texttt{Linux}:
        \begin{lstlisting}
            TEXINPUTS=.:/home/dimozzz/work/styles_tex/physsummer_style//:
        \end{lstlisting}
        при этом файл с переменными окружения находится в папке \texttt{/etc/} (в большинстве дистрибутивов);
    \item в операционной системе \texttt{Windows}:
        \begin{lstlisting}
            TEXINPUTS=.:C:\dimozzz\work\styles_tex\physsummer_style\\:
        \end{lstlisting}
\end{itemize}

В обоих случаях путь необходимо заменить на правильный. Первая точка означает, что компилятор всегда
проверяет текущую папку на предмет наличия необходимых файлов, два слеша в конце означает, что поиск
будет вестись также и по подпапкам указанной папки.