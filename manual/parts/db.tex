\section{Работа с <<базой>> задач \texttt{(physsummer-db)}}
\label{sec:db}

\subsection{<<База>> задач}
\label{sec:db-description}

Базой в данном случае мы будем называть папку, где хранятся условия задач, организованную специальным
способом. По умолчанию данная папка называется \texttt{problems\_db}, опишем её структуру.

\begin{lstlisting}[gobble = 3]
    problems_db/
        db.stamp
        zadachnik1/
            zadacha-1.tex
            zadacha-2.tex
            zadacha-2-fig.tex
        zadachnik2/
            zadacha-1.tex
            zadacha-2.tex
            info
        zadachnik3/
            zadacha-1-default.tex
            zadacha-1.tex
            zadacha-1-fig.tex
            zadacha-1-fig2.tex
            zadacha-2.tex
\end{lstlisting}

В корневой папке находится файл \texttt{db.stamp}, который необходим для корректной работы некоторых
команд. Также корневая папка содержит подпапки, с группами задач.

Задача без рисунка описывается одним файлом, в файле должен присутствовать только текст задачи, без
команды \texttt{task}.

Задача с рисунком~--- двумя или более файлами. В простом случае нужен файл с текстом задачи:
\texttt{zadacha.tex}, и файл с картинкой имеющий такое же имя, но с суффиксом \texttt{fig}:
\texttt{zadacha-fig.tex}. Если картинок более одной, то они должны иметь суффиксы \texttt{fig} с номерами
(см. пример).

В более сложном случае необходим файл с суффиксом \texttt{default}, в котором описывается расположение
текста, а также размер и расположение рисунков (см. раздел \ref{sec:db-default}).

\subsection{Настройка и использование базы}

Путь к базе указывается при помощи переменной \texttt{libproblempath} и по умолчанию значение этой
переменной указывает на текущую папку или на одну из папой не более, чем на четыре уровня выше текущей, в
которой был найден файл \texttt{db.stamp}. Если база задач находится в другом месте, то необходимо
вручную задать значение данной переменной при помощи следующей команды.

\begin{lstlisting}[gobble = 3]
    \renewcommand{\libproblempath}{правильный путь}
\end{lstlisting}

Ниже приведен пример использования задач из базы. Использована база из примера в разделе
\ref{sec:db-description}.


\begin{lstlisting}[gobble = 3]
    \begin{document}
        \libproblem{zadachnik-3}{zadacha-1}
        \libproblem{zadachnik-1}{zadacha-1}
        \libproblem{zadachnik-1}{zadacha-2}
    \end{document}
\end{lstlisting}




\subsection{Компоновка задачи: файл \texttt{default}}
\label{sec:db-default}

Компоновка задачи с картинкой по умолчанию выделяет $4\unit{см}$ для картинки. Для изменения компоновки
необходимо создать файл с суффиксом \texttt{default}.

В базовых случаях данный файл должен содержать одну единственную строку с одной из следующих команд.

\noindent
\begin{tabular}{S{0.48\linewidth} S{0.485\linewidth}}
    Изменение ширины картинки &
        \begin{lstlisting}[style = listtable, gobble = 10]
            \libproblemtaskpic[ширина картинки]
        \end{lstlisting} \\
    Отрисовка картинки под задачей &
        \begin{lstlisting}[style = listtable, gobble = 10]
            \libproblemtaskfigbelow
        \end{lstlisting} \\
    Расположение первой картинки на стандартном месте, а второй под задачей &
        \begin{lstlisting}[style = listtable, gobble = 10]
            \libproblemtaskdoublepic
        \end{lstlisting} \\
    Расположение двух картинок под задачей &
        \begin{lstlisting}[style = listtable, gobble = 10]
            \libproblemtasktwobelow
        \end{lstlisting}
\end{tabular}


В более сложных случаях может оказаться полезной одна из следующих команд.

\noindent
\begin{tabular}{S{0.48\linewidth} S{0.485\linewidth}}
    Извлечение картинки задачи &
        \begin{lstlisting}[style = listtable, gobble = 10]
            \libproblemfig[задачник][задача]
            % по умолчанию текущая задача
        \end{lstlisting} \\
    Извлечение текста задачи &
        \begin{lstlisting}[style = listtable, gobble = 10]
            \libproblemtext[задачник][задача]
            % по умолчанию текущая задача
        \end{lstlisting} \\
    Текущий задачник &
        \begin{lstlisting}[style = listtable, gobble = 10]
            \libproblemgroup
        \end{lstlisting} \\
    Текущая задача &
        \begin{lstlisting}[style = listtable, gobble = 10]
            \libproblemid
        \end{lstlisting} \\
    Создание метки для рисунка &
        \begin{lstlisting}[style = listtable, gobble = 10]
            \figrefname{#1}
            % сокращение для следующей команды
            \libproblemgroup:\libproblemid:#1
        \end{lstlisting} \\
\end{tabular}

Ниже приведён пример файла \texttt{default} со сложной нестандартной компоновкой.

\begin{lstlisting}
    \libproblemtaskpic % вставить текст задачи с первой картинкой
        % на стандартном месте
    \drawsidebyside{\figrefname{2}}{ % нарисовать две картинки на одном уровне
        \subimport{\libproblempath/\libproblemgroup/}{\libproblemid-fig2}
        % использовать вторую картинку
    }{\figrefname{3}}{
        \subimport{\libproblempath/\libproblemgroup/}{\libproblemid-fig3}
        % использовать третью картинку
    }
\end{lstlisting}