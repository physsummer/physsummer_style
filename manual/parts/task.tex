\section{Задачи \texttt{(physsummer-task)}}

Указанные в настоящем разделе команды зависят от пакета \texttt{tabularx}. При подключении дополнительных
инструментов обратите внимание на совместимость.

Пакет предоставляет две основные команды для набора задач.

\begin{lstlisting}[gobble = 4]
    \task{
        текст задачи
        \tags{ключевые слова}
        \answer{ответ задачи}
    }

    \taskpic[ширина картинки]{ % ширина является необязательным аргументом
        текст задачи           % и по умолчанию она равна 4 см
    }{
        картинка
    }
\end{lstlisting}

Команды \texttt{tags} и \texttt{answer} отображаются только в режиме учителя (см. раздел
\ref{sec:teacher}).


        
\subsection{Списки внутри задач}

Стандартный набор окружений \texttt{itemize} и \texttt{enumerate} не рекомендуется использовать в
задачах из-за больших отступов по умолчанию. Вместо этого пакет предоставляет следующие окружения,
которые реализованы посредством пакета \texttt{enumitem}.

\begin{lstlisting}[gobble = 3]
    \task{
        Список с маркерами:
        \begin{itemtask}
            \item $\vec{F} = m \vec{a}$;
            \item $PV = \nu RT$.
        \end{itemtask}
        Нумерованный список.
        \begin{enumtask} % 
            \item $\vec{F} = m \vec{a}$.
            \item $PV = \nu RT$.
        \end{enumtask}
        Список с русскими буквами:
        \begin{enumcyr}
            \item $\vec{F} = m \vec{a}$;
            \item $PV = \nu RT$.
        \end{enumcyr}
    }
\end{lstlisting}

\noindent
\task{
    Список с маркерами:
    \begin{itemtask}
        \item $\vec{F} = m \vec{a}$;
        \item $PV = \nu RT$.
    \end{itemtask}
    Нумерованный список.
    \begin{enumtask}
            \item $\vec{F} = m \vec{a}$.
            \item $PV = \nu RT$.
    \end{enumtask}
    Список с русскими буквами:
    \begin{enumcyr}
        \item $\vec{F} = m \vec{a}$;
        \item $PV = \nu RT$.
    \end{enumcyr}
}

\vspace{0.3cm}

Как видно в этом примере, в конце задачи присутствует пустая строка. К сожалению, это проблема
совместимости таблиц и списков в \LaTeX. Для того, чтобы избежать этого в пакете присутсвует команда
\texttt{lastline}, позволяющая исправить эту ситуацию. Также можно обратить внимание на стандартную
команду \texttt{normalbaselineskip}.

\begin{lstlisting}[gobble = 3]
    \task{
        Задача со списками.
        \begin{itemtask}
            \item $\vec{F} = m \vec{a}$;
            \item $PV = \nu RT$.
        \end{itemtask}\lastline
        И еще список.
        \begin{enumtask}
            \item $\vec{F} = m \vec{a}$.
            \item $PV = \nu RT$.
        \end{enumtask}\lastline
    }
\end{lstlisting}

\noindent
\task{
    Задача со списками.
    \begin{itemtask}
        \item $\vec{F} = m \vec{a}$;
        \item $PV = \nu RT$.
    \end{itemtask}\lastline
    И еще список.
    \begin{enumtask}
            \item $\vec{F} = m \vec{a}$.
            \item $PV = \nu RT$.
    \end{enumtask}\lastline
}

\setcounter{notask}{1}

\subsection{Частые ошибки}

В некоторых случаях таблицы недопускают пустых строк, поэтому может возникать ошибка компиляции, если в
тексте задачи присутствует пустая строка. В таком случае для переноса строки рекомендуется использовать
команду \texttt{newline} или \texttt{\textbackslash\textbackslash}. Данной ошибки не должно возникать при
работе с базой задач (см. раздел \ref{sec:db}).
