\section{Часто используемые команды \texttt{(physsummer-comm)}}

В данном разделе мы рассмотрим набор часто используемых команд, предоставляемых пакетом.

\subsection{Колонтитул}

Установка верхнего колонтитула:
\begin{lstlisting}[gobble = 3, extendedchars = \true]
    \setphysstyle{ЛФШ}{Мануал}{\texttt{physsummer}}
\end{lstlisting}

Данная команда применена к текущему документу.

\subsection{Математика}

\def\arraystretch{-500}

Часть команд из данного раздела предоставлена настоящим пакетом, часть пакетом \texttt{physics} и другими
(для дополнительных сведений см. мануал пакета \texttt{physics}).

\noindent
\begin{tabular}{C{0.28\linewidth} C{0.685\linewidth}}
    \vspace{-10pt}$\avg{\frac{a}{b}}$ &
        \begin{lstlisting}[style = listtable, gobble = 10]
            $\avg{\frac{a}{b}}$ % среднее значение
        \end{lstlisting} \\
    $\dd{x}, \dd[3]{x}$ &
        \begin{lstlisting}[style = listtable, gobble = 10]
            $\dd{x}, \dd[3]{x}$ %  дифференциал
        \end{lstlisting} \\
    \vspace{-10pt}$\dv{x}{t}, \dv[2]{x}{t}, \dv*{x}{t}$ &
        \begin{lstlisting}[style = listtable, gobble = 10]
            $\dv{x}{t}, \dv[2]{x}{t}, \dv*{x}{t}$  % производная
        \end{lstlisting} \\
    \vspace{-10pt}$\pdv{x}{t}, \pdv[2]{x}{t}, \pdv{f}{x}{y}$ &
        \begin{lstlisting}[style = listtable, gobble = 10]
            $\pdv{x}{t}, \pdv[2]{x}{t}, \pdv{f}{x}{y}$
            % частная производная
        \end{lstlisting} \\
    $\norm{a}$ &
        \begin{lstlisting}[style = listtable, gobble = 10]
            $\norm{a}$  % норма
        \end{lstlisting} \\
    $\const$ &
        \begin{lstlisting}[style = listtable, gobble = 10]
            $\const$ % констанста
        \end{lstlisting} \\
\end{tabular}



\subsection{Размерности}

\def\arraystretch{-500}

\noindent
\begin{longtable}{S{0.28\linewidth} C{0.685\linewidth}}
    C &
        \begin{lstlisting}[style = listtable, gobble = 10]
            $\unit{C}$  % команда размерности
        \end{lstlisting} \\
    $\cels$ &
              \begin{lstlisting}[style = listtable, gobble = 16]
                  $\cels$  % градусы Цельсия
              \end{lstlisting} \\
    $\djkgC$ &
               \begin{lstlisting}[style = listtable, gobble = 17]
                   $\djkgC$  % удельная теплоёмкость
               \end{lstlisting} \\
    $\kdjkgC$ &
                \begin{lstlisting}[style = listtable, gobble = 18]
                    $\kdjkgC$  % удельная теплоёмкость
                \end{lstlisting} \\
    $\kdjC$ &
              \begin{lstlisting}[style = listtable, gobble = 16]
                  $\kdjC$  % теплоёмкость
              \end{lstlisting} \\
    $\djkg$ &
              \begin{lstlisting}[style = listtable, gobble = 16]
                  $\djkg$  % удельная теплота
              \end{lstlisting} \\
    $\kdjkg$ &
               \begin{lstlisting}[style = listtable, gobble = 17]
                   $\kdjkg$  % удельная теплота
               \end{lstlisting} \\
    $\kgm$ &
             \begin{lstlisting}[style = listtable, gobble = 15]
                 $\kgm$  % плотность
             \end{lstlisting} \\
    $\gcm$ &
             \begin{lstlisting}[style = listtable, gobble = 15]
                 $\gcm$  % плотность
             \end{lstlisting} \\
    $\mc$ &
            \begin{lstlisting}[style = listtable, gobble = 14]
                $\mc$  % скорость
            \end{lstlisting} \\
    $\mcsq$ &
              \begin{lstlisting}[style = listtable, gobble = 16]
                  $\mcsq$  % ускорение
              \end{lstlisting} \\
    $\omm$ &
             \begin{lstlisting}[style = listtable, gobble = 15]
                 $\omm$  % удельное сопротивление
             \end{lstlisting} \\
\end{longtable}


\subsection{Счетчики}

В пакете используется ряд счётчиков. Для пользователя могут оказаться полезными счётчики:
\texttt{notask}~--- номер задачи, \texttt{trownumber}~--- нумерация колонок в таблицах (также см. раздел
\ref{sec:tables}).

\begin{lstlisting}[keepspaces, gobble = 3]
    \setcounter{notask}{1}
    \task{Задача 1.}
    \task{Задача 2.}
    
    \setcounter{notask}{1}
    \task{Снова задача 1.}
\end{lstlisting}

\setcounter{notask}{1}
\noindent
\task{Задача 1.}
\noindent
\task{Задача 2.}

\setcounter{notask}{1}
\noindent
\task{Снова задача 1.}

\setcounter{notask}{1}

\vspace{0.5cm}

Кроме того пакетом используется ряд внутренних счётчиков, для корректной работы пакета не используйте их:
\texttt{oldN}, \texttt{newN}, \texttt{jury}. Также присутствуют счётчики для комментариев (см. раздел
\ref{sec:authors}).


\subsection{Режим <<учителя>>}
\label{sec:teacher}

Путём добавления одной из двух команд.

\begin{lstlisting}[gobble = 3]
    \teachermode     % включение режима учителя
    \teachermodeoff  % выключение режима учителя
\end{lstlisting}

Работа некоторых команд может изменяться в зависимости от режима. Например, ответы к задачам показываются
только в режиме учителя.


\subsection{Работа с картинками}

\noindent
\begin{tabular}{S{0.48\linewidth} S{0.485\linewidth}}
    Отрисовка картинки по центру страницы &
        \begin{lstlisting}[style = listtable, gobble = 10]
            \drawfig{метка}{картинка}
        \end{lstlisting} \\
    Отрисовка картинки по центру страницы из файла &
        \begin{lstlisting}[style = listtable, gobble = 10]
            \figcap{файл}{ширина}
            % ширина задается долей от textwidth
        \end{lstlisting} \\
    Отрисовка картинки внутри текста &
        \begin{lstlisting}[style = listtable, gobble = 10]
            \drawwrap{метка}{ширина}{картинка}
            % ширина задается долей от textwidth
        \end{lstlisting} \\
    Отрисовка картинки внутри текста из файла &
        \begin{lstlisting}[style = listtable, gobble = 10]
            \wrapfigcap{файл}{ширина}
            % ширина задается долей от textwidth
        \end{lstlisting} \\
    Отрисовка картинок на одном уровне из файлов &
        \begin{lstlisting}[style = listtable, gobble = 10]
            \sidebyside{файл1}{ширина1}
                {файл2}{ширина2}
            % ширины задаются долей от textwidth
        \end{lstlisting} \\
    Отрисовка картинок на одном уровне &
        \begin{lstlisting}[style = listtable, gobble = 10]
            \drawsidebyside{метка1}{картинка1}
                {метка2}{картинка2}
                {ширина1}{ширина2}
            % ширины задаются долей от textwidth
            % ширины имеют значение по умолчанию 0.48
        \end{lstlisting} \\
\end{tabular}
