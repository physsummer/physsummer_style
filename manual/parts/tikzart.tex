\subsection{Зоопарк картинок \texttt{(physsummer-tikzart)}}

В картинках некоторых задач бывает необходимо нарисовать что-то более сложное.
К счастью, скорее всего это кто-то уже сделал за вас.


\begin{minipage}{0.28\linewidth}
    \begin{tikzpicture}[mstyle]
    \pic {tikzart-bunny};
    \end{tikzpicture}
\end{minipage}
\begin{minipage}{0.72\linewidth}
\begin{lstlisting}[gobble = 7]
        \begin{tikzpicture}
            \pic {tikzart-bunny};
            % pic по умолчанию рисует в точке (0, 0)
        \end{tikzpicture}
\end{lstlisting}
\end{minipage}

\begin{minipage}{0.28\linewidth}
    \begin{tikzpicture}[mstyle]
    	\draw[platform] (1.5, 0) -- (-2, 0);
        \pic[scale = 0.1] at (0.5, 0) {tikzart-cat};
        \pic[gray, scale = 0.125, xscale = -1] at (-1, 0) {tikzart-cat};
    \end{tikzpicture}
\end{minipage}
\begin{minipage}{0.72\linewidth}
    \begin{lstlisting}[gobble = 7]
        \begin{tikzpicture}
            \draw[platform] (1.5, 0) -- (-2, 0);
            \pic[scale = 0.1] at (0.5, 0) {tikzart-cat};
            % кошка в координатах (0.5, 0)
            \pic[gray, scale = 0.125, xscale = -1] at (-1, 0)
                {tikzart-cat};
                % рисуем серую кошку
                % размера 0.125 от исходного,
                % отражённую по горизонтали,
                % в точке (-1, 0)
        \end{tikzpicture}
    \end{lstlisting}
\end{minipage}


\begin{minipage}{0.28\linewidth}
    \begin{tikzpicture}[mstyle]
        \draw[platform] (2, 0) -- (-1, 0);
        \pic[scale = 0.65] {tikzart-dog};
        \begin{scope}[shift = {(0, -2)}]
            \draw[platform] (2, 0) -- (-2, 0);
            \pic[scale = 0.3, xscale = -1] {tikzart-frog};
            \pic[shift = {(0.5, 0)}, scale = 0.3] {tikzart-mouse};
        \end{scope}
    \end{tikzpicture}
\end{minipage}
\begin{minipage}{0.72\linewidth}
	\begin{lstlisting}[gobble = 7]
        \begin{tikzpicture}
            \draw[platform] (2, 0) -- (-1, 0);
            \pic[scale = 0.65] {tikzart-dog}; % собака
            
            \begin{scope}[shift = {(0, -2)}] % сдвигаемся на (0, -1.5)
                \draw[platform] (2, 0) -- (-2, 0);
                \pic[scale = 0.3, xscale = -1] {tikzart-frog};
                % лягушка

                \pic[scale = 0.3] at (0.5, 0) {tikzart-mouse}; % мышь
            \end{scope}
        \end{tikzpicture}
    \end{lstlisting}
\end{minipage}



\begin{minipage}{0.28\linewidth}
    \begin{tikzpicture}[mstyle]
        \draw[platform] (3, 0) coordinate(A)
            -| ++(-3, 3) coordinate (B);
        \coordinate (O) at ($(A-|B)$);
        \draw (B) ++(0, -0.5) pic[yscale = -1, rotate = -90] {tikzart-ladybug};
        \pic at ($(O)!0.75!(A)$) {tikzart-ant};
    \end{tikzpicture}
\end{minipage}
\begin{minipage}{0.72\linewidth}
    \begin{lstlisting}[gobble = 7]
        \begin{tikzpicture}
            \draw[platform] (3, 0) coordinate(A)
                -| ++(-3, 3) coordinate (B);
            \coordinate (O) at ($(A-|B)$);
            % коордниата (O) по горизонтали как (B) и вертикали как (A)

            \draw (B) ++(0, -0.5)
            % отступаем от координаты (B) на (0, -0.5)
                pic[yscale = -1, rotate = -90] {tikzart-ladybug};
                % рисуем божью коровку,
                % отражённую по вертикали

            \pic at ($(O)!0.75!(A)$) {tikzart-ant};
            % рисуем муравья на отрезке OA,
            % отступив на 0.75 длины от точки O
        \end{tikzpicture}
     \end{lstlisting}
\end{minipage}

\begin{minipage}{0.28\linewidth}
    \begin{tikzpicture}[mstyle]
        \pic[rotate = 90] {tikzart-cutter};
        \pic[rotate = 90] at (0, 1) {tikzart-car};
    \end{tikzpicture}
\end{minipage}
\begin{minipage}{0.72\linewidth}
    \begin{lstlisting}[gobble = 7]
        \begin{tikzpicture}
            \pic[rotate = 90] {tikzart-cutter};
            % катер, повёрнутый на 90 градусов
            
            \pic[rotate = 90] at (0, 1) {tikzart-car}; % машина
        \end{tikzpicture}
    \end{lstlisting}
\end{minipage}


\begin{minipage}{0.28\linewidth}
    \begin{tikzpicture}[mstyle]
        \draw[platform] (3, 0) -- (-2, 0);
        
        \draw[thick] (1.5, 0.1) -- (-1.5, 0.1);
        \foreach \x in {0, 1, 2}{
            \draw (-1.5 + \x, 0) pic[scale = 2] {tikzart-train};	
        }
        \pic[scale = 2] at (1.5, 0) {tikzart-trainhead};
        
        \draw (-1, 2) pic[scale = 0.75] {tikzart-plane};
    \end{tikzpicture}
\end{minipage}
\begin{minipage}{0.72\linewidth}
    \begin{lstlisting}[gobble = 7]
        \begin{tikzpicture}
            \draw[platform] (3, 0) -- (-2, 0);
        
            \draw[thick] (1.5, 0.1) -- (-1.5, 0.1);
            \foreach \x in {0, 1, 2}{ % рисуем вагоны в цикле
                \draw (-1.5 + \x, 0) pic[scale = 2] {tikzart-train};
            }
            \pic[scale = 2] at (1.5, 0) {tikzart-trainhead};
            % рисуем паровоз
        
            \draw (-1, 2) pic[scale = 0.75] {tikzart-plane}; % самолёт
        \end{tikzpicture}
\end{lstlisting}
\end{minipage}


\begin{minipage}{0.28\linewidth}
    \begin{tikzpicture}[mstyle]
        \draw[platform] (2, 0) -- (-1, 0);
        \pic[scale = 0.75, color = red, fill = red!50] {tikzart-fire};
        \pic[scale = 2] at (1.5, 1) {tikzart-dwarf};
    \end{tikzpicture}
\end{minipage}
\begin{minipage}{0.72\linewidth}
    \begin{lstlisting}[gobble = 7]
        \begin{tikzpicture}
            \draw[platform] (2, 0) -- (-1, 0);
            \pic[scale = 0.75, color = red, fill = red!50] {tikzart-fire};
            % огонь
            
            \pic[scale = 2] at (1.5, 1) {tikzart-dwarf}; % гном
        \end{tikzpicture}
    \end{lstlisting}
\end{minipage}

\begin{minipage}{0.28\linewidth}
    \begin{tikzpicture}[mstyle]
        \pic[scale = 0.6] at (-3.5, -5) {tikzart-gun};
    \end{tikzpicture}
\end{minipage}
\begin{minipage}{0.72\linewidth}
    \begin{lstlisting}[gobble = 7]
        \begin{tikzpicture}
            \pic[scale = 0.6] at (-3.5, -5) {tikzart-gun}; % ружьё
        \end{tikzpicture}
    \end{lstlisting}
\end{minipage}
